

\documentclass[paper=a4, fontsize=11pt]{article}
\usepackage{amsmath,amsfonts}

\title{
\textbf{Notes:} Inverse method for detecting subglacial lakes and sticky spots
}

\author{ Aaron Stubblefield (Columbia University)} % Your name

\date{\small\today} % Today's date or a custom date

\begin{document}

\maketitle % Print the title
\section*{Model derivation}
\noindent\textbf{Assumptions.}
\begin{itemize}
\item linear viscosity\\
\item linear sliding law
\end{itemize}
\vspace{0.2cm}
\noindent\textbf{Domain.}
\begin{itemize}
\item The unperturbed domain is the infinite strip $\{(x,y,z)\in\mathbb{R}\times\mathbb{R}\times[s,h]\}$ for some constant
elevations $s$ and $h$.
\end{itemize}
\noindent\textbf{Body equations.}\\
Linear stokes flow:
\begin{align}
&-p_x + \eta ( u_{xx} +u_{yy} + u_{zz}) = 0  \\
&-p_y + \eta ( v_{xx}+v_{yy} + v_{zz}) = 0 \\
&-p_z + \eta ( w_{xx} +w_{yy} + w_{zz}) = \rho g \\
&u_x + v_y + w_z = 0
\end{align}
\\
\noindent\textbf{Upper surface BC ($z=h$).}\\
Stress-free conditions:
\begin{align}
&2\eta w_z - p = 0 \\
&\eta(u_z +w_x) = 0 \\
&\eta(v_z +w_y) = 0 \\
\end{align}
Kinematic equation:
\begin{align}
h_t + uh_x + vh_y = w
\end{align}

\noindent\textbf{Lower surface BC ($z=s$).}\\
Prescribed vertical velocity and a linear sliding law:
\begin{align}
&w  = w_b\\
&\eta(u_z + w_x) = \alpha u \\
&\eta(v_z + w_y) = \alpha v
\end{align}
\noindent Kinematic equation:
\begin{align}
s_t + us_x + vs_y  = w
\end{align}
\noindent\textbf{Equilibrium solution:}\\
We assume the equilibrium solution is either cryostatic or plug flow in the $x$ direction.\\
Therefore, we choose $h^0=H$ and $s^0=0$ constant, and...
\begin{align}
&u^0 = u_0\geq 0\;\;\;\; \text{(constant)} \\
& v^0 = 0 \\
&w^0 = 0 \;\;\;\; (w_b^0=0)\\
&p^0 = \rho g (H-z) \\
&\alpha^0  = \begin{cases}
\alpha_0 = 0 & \text{if} \;\; u_0>0 \\
\alpha_0 \approx \infty & \text{if} \;\; u_0 =0
\end{cases}
\end{align}


\noindent\textbf{Perturbations:}\\
We introduce perturbations to the equilibrium state via
\begin{align}
&u= u^0 +  u^1 \\
&v= v^0 +  v^1 \\
&w =  w^0 + w^1 \;\; (w_b=w_b^0 + w_b^1)\\
&p = p^0 +  p^1\\
&s =  s^0 + s^1 \\
&h = H +  h^1
\end{align}
where the perturbations are small ($O(\epsilon)$ where $\epsilon \ll 1$). \\ \\
\noindent\textbf{Perturbed equations:}\\
Method: insert perturbations and discard product terms (i.e., $a^1b^1=O(\epsilon^2)$).\\
Body equations become:
\begin{align}
&-p_x^1 + \eta (u_{xx}^1 +u_{yy}^1+ u_{zz}^1) = 0 \\
&-p_x^1 + \eta (v_{xx}^1 +v_{yy}^1+ v_{zz}^1) = 0 \\
&-p_z^1 + \eta (w_{xx}^1 +w_{yy}^1+ w_{zz}^1) = 0\\
&u_x^1 + u_y^1 + w_z^1 = 0
\end{align}
Surface kinematic equations become:
\begin{align}
&h_t^1 + u_0 h_x^1  = w^1 \label{ht} \\
&s_t^1  + u_0 s_x^1 = w^1 \label{st}
\end{align}
We have to linearize the upper and lower surface equations at $z=H +  h^1$
and $z= s^1$ onto $z=H$ and $z=0$.
\\ \\
To do this, we
use the $1^\mathrm{st}$-order Taylor expansion in depth for a function $f(z)$:
$$ f(z^0 + z^1) \approx f(z^0) + f'(z^0)z^1. $$
\\
The stress-free condition at $z=H+h^1$ is approximated at $z=H$ by
\begin{align}
&2\eta w_z^1 - p^1 = -\rho g h^1 \label{pnorm} \\
&\eta(u_z^1 +w_x^1) = 0 \\
&\eta(v_z^1 +w_y^1) = 0
\end{align}
Equation (\ref{pnorm}) states that the perturbed normal stress is balanced
by the perturbed cryostatic stress from the elevation anomaly.
\\ \\
BC's at the lower boundary ($z=0$) become:
\begin{align}
&w^1  = w_b^1\\
&\eta(u_z^1 + w_x^1) = \alpha_0 u^1 + u_0\alpha^1 \\
&\eta(v_z^1 +w_y^1) = \alpha_0 v^1
\end{align}
We drop the superscripts below. \\ \\
\\ \noindent\textbf{Fourier transform approach:}\\
The 2D Fourier transform of a function $f$ with respect to $x$ and $y$ is
\begin{align}
\widehat{f}(k_x,k_y)  = \int_{-\infty}^{+\infty}\int_{-\infty}^{+\infty} f(x,y)e^{-i(k_x x + k_y y)} \; \mathrm{d}x\,\mathrm{d}y,
\end{align}
where $k_x$ and $k_y$ are wavenumbers.
The inverse Fourier transform is given by
\begin{align}
{f}(x,y) = \frac{1}{4\pi^2}\int_{-\infty}^{+\infty}\int_{-\infty}^{+\infty} \widehat{f}(k_x,k_y)e^{i(k_x x + k_y y)} \; \mathrm{d}k_x \, \mathrm{d}k_y.
\end{align}
Derivatives transform as:
\begin{align}
&\widehat{f_x} = ik_x\widehat{f} \\
&\widehat{f_y} = ik_y\widehat{f} \\
&\widehat{f_{xx}} + \widehat{f_{yy}} = -k^2 \widehat{f}, \\ &k^2 \equiv k_x^2 + k_y^2.
\end{align}
The upper and lower surface kinematic equations (\ref{ht})-(\ref{st}) transform to
\begin{align}
\widehat{h}_t + u_0 ik_x \widehat{h} = \widehat{w}\label{hthat}\\
\widehat{s}_t + u_0 ik_x \widehat{s} = \widehat{w}.\label{sthat}
\end{align}
The Stokes flow equation become
\begin{align}
&-ik_x\widehat{p} + \eta ( -k^2\widehat{u} + \widehat{u}_{zz}) = 0 \\
&-ik_y\widehat{p} + \eta ( -k^2\widehat{v} + \widehat{v}_{zz}) = 0 \\
&-\widehat{p}_z + \eta (-k^2\widehat{w} + \widehat{w}_{zz}) = 0 \\
&ik_x\widehat{u} + ik_y\widehat{v} + \widehat{w}_z = 0
\end{align}
This can be reduced to a fourth-order equation for the transformed vertical velocity:
\begin{equation}
\widehat{w}_{zzzz} - 2k^2 \widehat{w}_{zz} + k^4 \widehat{w}=0. \label{ode}
\end{equation}
The general solution to (\ref{ode}) is
\begin{align}
\widehat{w} = \frac{A}{k}e^{k z} + \frac{B}{k}e^{-k z} + {C}ze^{k z}+ {D}ze^{-k z},
\end{align}
where $A,B,C,$ and $D$ depend on $k$.\\ \\
To determine the coefficients, we rewrite all the BC's in terms of $\widehat{w}$ and its $z$ derivatives.
The $z$-derivatives of $\widehat{w}$ are:
\begin{align}
&\widehat{w}_{z} = {A}e^{k z} - {B}e^{-k z} + {C}e^{k z} + {C}kze^{k z} - {D}kze^{-k z} + {D}e^{-k z} \\
&\widehat{w}_{zz} = {Ak}e^{k z} + {Bk}e^{-k z} + {2Ck}e^{k z} + {C}k^2 ze^{k z} + {D}k^2ze^{-k z} - {2Dk}e^{-k z} \\
&\widehat{w}_{zzz} = {Ak^2}e^{k z} - {Bk^2}e^{-k z} + {3Ck^2}e^{k z} + {C}k^3 ze^{k z} - {D}k^3ze^{-k z} + {3Dk^2}e^{-k z}
\end{align}
The sliding law becomes
\begin{align}
&\eta( \widehat{u}_z + ik_x \widehat{w}) = \alpha_0\widehat{u} + u_0\widehat{\alpha} \\
&\eta( \widehat{v}_z + ik_y \widehat{w}) = \alpha_0\widehat{v}.
\end{align}
Multiplying through by $-ik_x$ and $-ik_y$, summing the equations, and using the transformed incompressibility
condition reduces this to
\begin{align}
\eta(\widehat{w}_{zz} + k^2 \widehat{w}) = \alpha_0 \widehat{w}_z - ik_x   u_0\widehat{\alpha}.
\end{align}
Similarly, the shear stress condition at the upper surface becomes
\begin{align}
\eta(\widehat{w}_{zz} + k^2 \widehat{w}) =  0.
\end{align}
The normal stress condition at the upper surface transforms to
$
2\eta \widehat{w}_{z} - \widehat{p} = -\rho g \widehat{h}.
$
From the body equations, we can show that
$-k^2 \widehat{p} = \eta(k^2\widehat{w}_z - \widehat{w}_{zzz}) $, which reduces this to
\begin{align}
\eta (3k^2 \widehat{w}_{z}-\widehat{w}_{zzz})  = -k^2 \rho g \widehat{h}.
\end{align}
To simplify notation, we define the `aspect ratio' $\nu(k) = kH$.\\ \\
The upper surface normal stress BC becomes:
\begin{align}
{A} e^{\nu} - {B} e^{-\nu} + {C}\nu e^{\nu} - {D}\nu e^{-\nu}
=- \frac{\rho g }{2\eta }\widehat{h} \label{b1}
\end{align}
The upper surface shear-stress BC becomes:
\begin{align}
A e^{\nu} + B e^{-\nu} + C(\nu+1) e^{\nu} +D(\nu-1) e^{-\nu} =  0
\end{align}
The sliding law BC becomes:
\begin{align}
{A(1-\gamma)} + {B(1+\gamma)} + {C(1-\gamma)}  - {D(1+\gamma)} = -\frac{ ik_x }{2\eta k}u_0\widehat{\alpha}
\end{align}
where we have defined $\gamma(k) = \alpha_0/(2\eta k)$. \\
\\ The basal velocity anomaly BC becomes
\begin{align}
{A} + {B}  = k\widehat{w}_b \label{b4}
\end{align}
Equations (\ref{b1})-(\ref{b4}) lead to a linear system
...which we solve with SymPy.\\ \\
We can write the anomaly at the surface as
\begin{align}
\widehat{w}|_{z=H} &= \frac{1}{k}\left(e^{\nu} A + { e^{-\nu}}B + \nu e^{\nu} C + {\nu}{ e^{-\nu}} D\right) \\
&= -\mathcal{R}\widehat{h} + \mathcal{T}_w\widehat{w}_b  +ik_x u_0  \mathcal{T}_{\alpha}  \widehat{\alpha}\label{wRT}
\end{align}
According to SymPy, the relaxation frequency $\mathcal{R}$ is given by
\begin{align}
\mathcal{R}(k,\nu,\gamma) = \left(\frac{\rho_i g }{2\eta k}\right)\frac{ (1+\gamma)e^{4\nu} -(2+4\gamma\nu)e^{2\nu} +1-\gamma  }{ (1+\gamma)e^{4\nu} + (2\gamma+4\nu+4\gamma\nu^2)e^{2\nu} -1 + \gamma  },
\end{align}
and velocity anomaly transfer function $\mathcal{T}_w$ is given by
\begin{align}
\mathcal{T}_w(\nu,\gamma) = \frac{2(1+\gamma)(\nu+1)e^{3\nu}+2(1-\gamma)(\nu-1)e^{\nu}  }{(1+\gamma)e^{4\nu} + (2\gamma+4\nu+4\gamma\nu^2)e^{2\nu} -1 + \gamma },
\end{align}
and the friction perturbation transfer function is given by
\begin{align}
\mathcal{T}_{\alpha}(k,\nu) =  \left(\frac{\nu}{\eta k^2}\right) \frac{e^{3\nu} + e^{\nu}}{e^{4\nu} +4\nu e^{2\nu} -1 }
\end{align}
where we have assumed $\gamma=0$  since $u_0>0$ (otherwise the term vanishes).
 \\ \\ \newpage
\noindent\textbf{Some limits of $\mathcal{R}$ and $\mathcal{T}_w$.}\\
\begin{itemize}
\item \emph{Infinite ice thickness (${\nu}\to\infty$ with k fixed):}
\begin{align}
&\lim_{\nu\to\infty} \mathcal{R}({k},\nu,{\gamma}) = \frac{\rho_i g }{2\eta {k}} \\
&\lim_{\nu\to\infty} \mathcal{T}_w(\nu,{\gamma}) =0
\end{align}
For harmonic perturbations, this results in the classical viscous
relaxation solution given in Turcotte \& Schubert's \emph{Geodynamics} textbook.
\item \emph{Vanishing ice thickness (${\nu}\to 0$ with k fixed):}
\begin{align}
&\lim_{\nu\to 0} \mathcal{R}({k},\nu,{\gamma}) = 0  \\
&\lim_{\nu\to 0} \mathcal{T}_w(\nu,{\gamma}) = 1
\end{align}
In this limit, the surface anomaly corresponds perfectly with the basal anomaly
and there is no viscous relaxation.
\item \emph{No-slip at bed (${\gamma}\to \infty$):}
\begin{align}
\lim_{\gamma\to\infty} \mathcal{T}_w({\nu},{\gamma}) = \frac{2(1+\nu) e^{\nu} + 2 (1-\nu) e^{-\nu} }{ 2+4\nu^2  + e^{2\nu} +e^{-2\nu}}
\end{align}
This is the transfer function (eqn. 21b) from Balise \& Raymond (1985) where no-slip is considered.
\end{itemize}

\section*{General solutions.}
In frequency space,
equations (\ref{wRT}) and (\ref{hthat}) lead to the evolution equation
\begin{align}
\frac{\partial \widehat{h}}{\partial t}+ \left[ik_xu_0  + \mathcal{R}\right]\widehat{h} = \mathcal{T}_w\widehat{w}_b+ik_xu_0\mathcal{T}_{\alpha} \widehat{\alpha}. \label{dhhat}
\end{align}
\\
Here we introduce a scaling for (\ref{dhhat}).
The goal is to rewrite the equations in terms of the nondimensional
aspect ratio $\nu$.
We let $h_0$ be a measure of the elevation anomaly magnitude and
$t_p$ a measure of the filling-draining timescale (e.g., oscillation period).
We introduce the following scaling and definitions
\begin{align}
& h = h_0 h', \;\;\;\; s = h_0 s',\;\;\;\; w_b = w_0 w_{b} , \;\; w_0 = h_0 t_p^{-1} \\
& x = Hx', \;\;\;\;
t = t_p t', \;\;\;\;\\
& k' = \nu = kH , \;\;\;\; k_x' = k_x H,\;\;\;\; k_y' = k_y H
\end{align}
We scale the relaxation function as
\begin{align}
   &\mathcal{R} = t_r^{-1} \mathcal{R}', \;\;\;\;
 \mathcal{R}'(\nu) =  \frac{2}{\nu}\frac{ (1+\gamma)e^{4\nu} -(2+4\gamma\nu)e^{2\nu} +1-\gamma  }{ (1+\gamma)e^{4\nu} + (2\gamma+4\nu+4\gamma\nu^2)e^{2\nu} -1 + \gamma  }
\end{align}
where
\begin{align}
t_r \equiv \frac{4\eta}{\rho_i g H}
\end{align}
is the characteristic relaxation time for perturbations with $\pi H$ wavelength.
The background friction-viscosity ratio $\gamma=\alpha_0/(2\eta k)$ can be written as
\begin{align}
   &\gamma = \alpha_0' \nu^{-1}, \;\;\;\; \alpha_0' = \frac{\alpha_0 H}{2\eta}.
\end{align}
Similarly, we scale the friction perturbation as
\begin{align}
\alpha' = \frac{ \alpha H}{2\eta}.
\end{align}
The scaled friction transfer function is
\begin{align}
&\mathcal{T}_{\alpha}'(\nu) = \left(\frac{2}{\nu}\right)\frac{e^{3\nu} + e^{\nu}}{e^{4\nu} +4\nu e^{2\nu} -1 }.
\end{align}
Note that $\mathcal{T}_w=\mathcal{T}_w'$ is already nondimensional, depending only on $k' = \nu$.
With this scaling, $0\leq \mathcal{R}' \leq 1$ and $0\leq \mathcal{T}_w' \leq 1$  . \\ \\
Dropping primes, we can rewrite the system as
\begin{align}
\frac{\partial \widehat{h}}{\partial t} +\left[ik_x \bar{u} + \lambda\mathcal{R} \right]\widehat{h} = \mathcal{T}_w\widehat{w}_b +i k_x \bar{u} \mathcal{T}_{\alpha}\widehat{\alpha}, \label{dhhatsc}
\end{align}
where
\begin{align}
  \lambda \equiv  \frac{t_p}{t_r}
\end{align}
is the filling-draining time relative to the relaxtion time and
\begin{align}
\bar{u} \equiv \frac{u_0 t_p}{H} = \left(\frac{u_0}{H}\right) / \left(\frac{w_0}{h_0}\right)
\end{align}
is the strain rate scale from the background flow relative to the strain rate scale from the basal anomaly.
\\ \\
Assuming zero initial displacement, the solution for the transformed elevation anomaly is
\begin{align}
\widehat{h} = (\mathcal{T}_w\widehat{w}_b + ik_x\bar{u}\mathcal{T}_{\alpha}\widehat{\alpha}  )* \mathcal{K} \label{hath0}
\end{align}
where $*$ denotes convolution with respect to time $t$ and
the flow/relaxation kernel $\mathcal{K}$ is
\begin{align}
\mathcal{K}(t) &= e^{-[ik_x\bar{u}+\lambda \mathcal{R}]t}
\end{align}
with modulus $|\mathcal{K}| = e^{-\lambda\mathcal{R}t}$.
\\ \\
Some limiting behavior is immediately apparent:
\begin{itemize}
\item \emph{Fast oscillation} ($\lambda \to 0$):  \\ In this case, we have
$\widehat{h} = \mathcal{T}_w\widehat{s} $.
Therefore, $\widehat{h}$ evolves as some fraction of $\widehat{s}$ (since $\mathcal{T}_w\leq 1$) with no phase lag.
\item \emph{Slow oscillation} ($\lambda\to \infty$): \\
In this case, $\widehat{h}=0$. Upper surface is not influenced by lower surface.
\item \emph{Fast flow} ($\bar{u}\to\infty$) with free slip ($\widehat{\beta}=0$): \\ The kernel
$\mathcal{K}$ becomes highly oscillatory. Therefore, $\widehat{h}\to 0$ due to the Riemann-Lebesgue lemma.
\end{itemize}
The general solution is then obtained as the inverse Fourier transform of (\ref{hath0}).

\section*{Inverse problem}
We let ${h}^{\mathrm{obs}}(x,t)$ be the the observed elevation anomaly with $t\in[0,T]$.
The forward operator that maps $(w_b,\alpha)$ to the elevation $h$ is
\begin{align}
\mathsf{A}(w_b,\alpha) = \mathsf{F}^{-1} \left( \mathsf{F}({w}_b)*(\mathcal{T}_w \mathcal{K})\right) + \mathsf{F}^{-1}\left( \mathsf{F}(\alpha) * (ik_x\bar{u}\mathcal{T}_{\alpha}\mathcal{K})\right),
\end{align}
where $\mathsf{F}$ and $\mathsf{F}^{-1}$ denote the forward and inverse Fourier transforms.
For now, we will focus on the case $w_b=0$ or $\alpha=0$, so that the
solution operator takes the form
\begin{align}
\mathsf{A}(f) = \mathsf{F}^{-1} \left( \mathcal{K}_0* \mathsf{F}(f)\right)
\end{align}
where $f=w_b$ (or $\alpha$) and $\mathcal{K}_0 =\mathcal{T}_w \mathcal{K} $ (or $ik_x\bar{u}\mathcal{T}_{\alpha}\mathcal{K}$).
The least-squares solution is found by minimizing the functional
\begin{align}
\mathsf{J}(f) = \langle \mathsf{A}(f) - h^{\mathrm{obs}} ,\mathsf{A}(f) - h^{\mathrm{obs}} \rangle + \mathsf{R}(f) \label{lsq}
\end{align}
where we use the inner product
\begin{align}
\langle f,g\rangle = \int_0^T \int_{-\infty}^{+\infty} \int_{-\infty}^{+\infty} fg \;\mathrm{d}x\,\mathrm{d}y\,\mathrm{d}t,
\end{align}
which is associated with the norm
\begin{align}
\| f \| = \sqrt{\langle f,f\rangle}.
\end{align}
The adjoint of $\mathsf{A}$, i.e. the
operator $\mathsf{A}^\dagger$  satisfying $\langle \mathsf{A}(f_1),f_2 \rangle = \langle f_1, \mathsf{A}^\dagger(f_2) \rangle$,
is
\begin{align}
\mathsf{A}^\dagger (f) = \mathsf{F}^{-1} \left( \mathcal{K}_0 \star \mathsf{F}(f)\right),
\end{align}
where $\star$ denotes the cross-correlation operator.
This can be proven by noting that $\mathsf{F}$ is the adjoint of $\mathsf{F}^{-1}$ (w.r.t. the $x$ and $y$ integrals)
and $\star$ is the adjoint of $*$ (w.r.t. the $t$ integral).\\ \\
Supposing that the first variation of (\ref{lsq}) vanishes, we obtain the
(infinite-dimensional) normal equations
\begin{align}
\mathsf{A}^\dagger (\mathsf{A}(f)) + \delta\mathsf{R}(f) = \mathsf{A}^\dagger (h^{\mathrm{obs}}), \label{normal}
\end{align}
where $\delta\mathsf{R}$ denotes the variational derivative of $\mathsf{R}$.
Provided that $\delta\mathsf{R}$ is linear in $f$, we can solve (\ref{normal})
with the conjugate gradient method.

\end{document}
